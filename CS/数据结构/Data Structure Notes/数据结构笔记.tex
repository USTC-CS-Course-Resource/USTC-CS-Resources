\documentclass[UTF8]{article}
\usepackage{amsmath}
\usepackage{makecell}
\usepackage[utf8]{inputenc}
\usepackage[space]{ctex} %中文包
\usepackage{listings} %放代码
\usepackage{xcolor} %代码着色宏包
\usepackage{CJK} %显示中文宏包

\definecolor{mygreen}{rgb}{0,0.6,0}
\definecolor{mygray}{rgb}{0.5,0.5,0.5}
\definecolor{mymauve}{rgb}{0.58,0,0.82}
\lstset{
	backgroundcolor=\color{white}, 
	basicstyle = \footnotesize,       
	breakatwhitespace = false,        
	breaklines = true,                 
	captionpos = b,                    
	commentstyle = \color{mygreen}\bfseries,
	extendedchars = false,
	frame =shadowbox, 
	framerule=0.5pt,
	keepspaces=true,
	keywordstyle=\color{blue}\bfseries, % keyword style
	language = C++,                     % the language of code
	otherkeywords={string}, 
	numbers=left, 
	numbersep=5pt,
	numberstyle=\tiny\color{mygray},
	rulecolor=\color{black},         
	showspaces=false,  
	showstringspaces=false, 
	showtabs=false,    
	stepnumber=1,         
	stringstyle=\color{mymauve},        % string literal style
	tabsize=4,          
	title=\lstname                      
}


%画图包
\usepackage{tikz}
%画图背景包
\usetikzlibrary{backgrounds}

%自定义命令
\newcommand{\psiG}{\psi_{G}}
%在tikz中画一个顶点
%#1:node名称
%#2:位置
%#3:标签
\newcommand{\newVertex}[3]{\node[circle, draw=black, line width=1pt, scale=0.8] (#1) at #2{#3}}
%在tikz中画一条边
\newcommand{\newEdge}[2]{\draw [black,very thick](#1)--(#2)}
%在tikz中放一个标签
%#1:名称
%#2:位置
%#3:标签内容
\newcommand{\newLabel}[3]{\node[line width=1pt] (#1) at #2{#3}}
\newcommand{\jumpLine} {\hspace*{\fill} \par}

\newcommand{\keypoint}[2]{$\bullet$\textbf{#1}\quad#2\par}
\newcommand{\vr}{\bm{r}}
\newcommand{\average}[1]{\left\langle #1\right\rangle }
\newcommand{\tabincell}[2]{\begin{tabular}{@{}#1@{}}#2\end{tabular}}%放在导言区
\newcommand{\points}[1]{
	$$ \left\{ 
	\begin{array}{l}
	#1
	\end{array}
	\right.
	$$
}



\title{计算方法编程练习:级数求和}
\author{王章瀚 PB18111697}


\begin{document}
\section{线性表}
26页“时间复杂度却不同”\\



\section{串}


	
	
	
	
\section{树}
\subsection{树的存储结构}
\subsubsection{双亲表示法}
如果需要访问双亲的情况比较多,可以用这种方法。\par
不仅可以用来表示数,也可以用来表示森林。(显然的。不连通即可。)
\subsubsection{孩子表示法}
容易访问孩子,不容易访问双亲。
需要增加头结点去表示森林
\begin{lstlisting}[language=C++]
//孩子链表节点
class CTNode{
	int child;
	CTNode *next;
};
tpyedef CTNode* ChildPtr;

//孩子链表
class CTBox{
	TElemType data;
	//孩子链表头指针
	ChildPtr firstchild;
};

//孩纸表示法树
class CTree{
	CTBox nodes[MAX_TREE_SIZE];
	int n, r;
}
\end{lstlisting}
\subsubsection{兄弟表示法}
可以把所有根节点当作兄弟。此时可以对森林作先根遍历(对应二叉树先序遍历)和后根遍历(对应二叉树中序遍历)。
\begin{lstlisting}[language=C++]
//孩子兄弟链表节点
class CSNode{
	ElemType data; //节点的数据
	CSNode *firstchild; //节点的第一个孩子
	CSNode *nextsibling; //节点的兄弟
};
\end{lstlisting}
\subsubsection{例8}
统计树(森林)中叶子节点个数。即二叉链中结点的firstchild为空的数目。
\begin{lstlisting} [language=C++]
int n = 0;
void CSleaf(CSTree T){
	if(T != nullptr]) {
		if(T->firstchild != nullptr) n++;
		CSleaf(T->firstchild);
		CSleaf(T->nextsibling);
	}
}
\end{lstlisting}

\subsubsection{例10}
求树的度(最大分支数)
\begin{lstlisting} [language=C++]
void SCDegree(CSTree T, int &degree) {
	if(T == NULL) degree = 0;
	else {
		d = 1;p = T->firstchild;
		// 遍历兄弟来统计孩子数
		while(p->nextsibling != NULL) {
			p = p->nextsibling; d++;
		}
	}
	CSDegree(T->firstchild, d1); CSDegree(T->nextsilbling, d2);
	degree = max(d1, d2, d);
}
\end{lstlisting}

\subsection{最优树,最优二叉树——赫夫曼树}
\subsubsection{一些概念}
路径:从树中一个结点到另一个结点之间的分支。\par
路径长度:路径分支数目。\par
树的路径长度:根到每一结点路径长度之和\par
结点的带权路径长度:该结点到树根之间的路径长度与结点上权的乘积\par
树的带权路径长度:加起来。$WPL=\sum_{i=1}^{n}w_{i}*l_{i}$
\subsubsection{赫夫曼树的应用}
最佳判定算法。

\subsection{树与等价问题}
$\bullet$等价类划分\par
$\bullet$MFSet树型结构,Find(S, x),Merge(\&S, i, j)\par

\subsection{回溯法与树的遍历}
\subsubsection{求含n个元素的集合的幂集}
算法基本描述如下:\par
$\bullet$n个元素$\rightarrow$高度为n的满二叉树\par
$\bullet$根节点:空集\par
$\bullet$叶子:形成幂集中的一个元素\par
$\bullet$左分支:取\par
$\bullet$右分支:舍\par

\subsubsection{树的计数}
具有n个节点的不同形态的树有多少棵?\par
值相同则称\textbf{等价},不同则称\textbf{相似}\par
给定先序序列,寻找中序遍历的所有可能性。\par


\section{图}
$\bullet$图在邻接矩阵和邻接表上的遍历算法及应用。\par
$\bullet$求图的最小生成树,最短路径,拓扑排序,关键路径等算法及其应用,性能分析。\par
一些定义:
$\bullet$无向图:\par
$\qquad$$\bullet$无向图边:边$(v,w)\in E$,则$v$与$w$为邻接点\par
$\qquad$$\bullet$无向图度:$TD(v)$\par
$\qquad$$\bullet$连通分量:极大连通子图\par

$\bullet$有向图:\par
$\qquad$$\bullet$有向图弧:弧$\left< v,w\right>\in E$,$w$为弧头,$v$为弧尾。顶点$v$邻接到顶点$w$,顶点$w$邻接自顶点$v$\par
$\qquad$$\bullet$有向图度:入度:$ID(v)$;出度$OD(v)$\par
$\qquad$$\bullet$简单路径:没有环路的路径\par
$\qquad$$\bullet$简单环:没有重复绕圈圈\par
$\qquad$$\bullet$强连通图:可以有向连通的有向图\par
$\qquad$$\bullet$强连通分量:有向图中的极大强连通子图\par

$\bullet$稀疏图:$e<nlogn$\par
$\bullet$网:图中边或弧上有权。有向图,有向网,无向图,无向网\par
\subsection{关系集存储方法}
教材算法7.1-7.3
\subsubsection{数组表示法邻接矩阵}
\begin{lstlisting} [language=C++]
typedef struct {
	// 顶点向量
	VertexType vexs[MAX_VERTEX_NUM];
	AdjMatrix arcs;
	int vexnum, arcnum;
	GraphKind kind;
}
\end{lstlisting}

\subsubsection{邻接表}
\begin{lstlisting} [language=C++]
typedef struct ArcNode { //表结点
	int adjvex; //弧指向的顶点的位置
	struct ArcNode *nextarc; 指向下一条弧的指针
	InfoType *info;
} ArcNode;
\end{lstlisting}

\subsection{有向无环图及其应用}
\keypoint{表达式共享子式}{}
\keypoint{描述一项工程}{工程由若干\textbf{活动}组成}
\subsubsection{拓扑排序}
\keypoint{顶点表示活动的网(AOV-网)}{用弧表示活动间有限关系的有向图}
\keypoint{拓扑排序算法}{
	\points{
		\mbox{1.在图中找一个没有前驱的顶,输出;}\\
		\mbox{2.删除以其为尾的弧,直至所有顶已输出.可建立一个0度顶栈来实现.}\\
		\mbox{时间复杂度为o(n+e)}\\
	}
}
\subsubsection{关键路径}



	
	
	
	
\end{document}
