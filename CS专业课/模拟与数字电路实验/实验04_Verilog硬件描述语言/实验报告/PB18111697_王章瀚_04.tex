\documentclass[UTF8]{article} 
\usepackage{graphicx}
\usepackage{subfigure}
\usepackage{amsmath}
\usepackage{makecell}
\usepackage[utf8]{inputenc}
\usepackage[space]{ctex} %中文包
\usepackage{listings} %放代码
\usepackage{xcolor} %代码着色宏包
\usepackage{CJK} %显示中文宏包
\usepackage{float}


\definecolor{mygreen}{rgb}{0,0.6,0}
\definecolor{mygray}{rgb}{0.5,0.5,0.5}
\definecolor{mymauve}{rgb}{0.58,0,0.82}
\lstset{
	backgroundcolor=\color{white}, 
	basicstyle = \footnotesize,       
	breakatwhitespace = false,        
	breaklines = true,                 
	captionpos = b,                    
	commentstyle = \color{mygreen}\bfseries,
	extendedchars = false,             
	frame =shadowbox, 
	framerule=0.5pt,
	keepspaces=true,
	keywordstyle=\color{blue}\bfseries, % keyword style
	language = Verilog,                     % the language of code
	otherkeywords={string}, 
	numbers=left, 
	numbersep=5pt,
	numberstyle=\tiny\color{mygray},
	rulecolor=\color{black},         
	showspaces=false,  
	showstringspaces=false, 
	showtabs=false,    
	stepnumber=1,         
	stringstyle=\color{mymauve},        % string literal style
	tabsize=4,          
	title=\lstname                      
}


\title{中国科学技术大学计算机学院\\《数字电路实验》报告}
\author{}

\date{}

\begin{document}
	\maketitle
	\begin{figure}[H]
		\centering
		\includegraphics[width=2.5in]{xiaohui.jpg}\vspace{0.5cm}\\
		\large{
			实验题目:Verilog硬件描述语言\\
			学生姓名:王章瀚\\
			学生学号:PB18111697\\
			完成日期:2019/10/25\\
		}\vspace{2cm}
		
		\large{计算机实验教学中心制\\2019年09月\\}
		\thispagestyle{empty}
		\clearpage  % 清除当页页码
	\end{figure}


	\section{实验目的}
	掌握 Verilog HDL 常用语法\par
	能够熟练阅读并理解 Verilog 代码\par
	能够设计较复杂的数字功能电路\par
	能够将 Verilog 代码与实际硬件相对应\par
	
	\section{实验环境}
	PC 一台\par
	Windows 或 Linux 操作系统\par
	Java 运行环境(jre)\par
	Logisim 仿真工具\par
	vlab.ustc.edu.cn (jre 和 Logisim 工具都可在此网站获取)\par
	
	\section{实验过程}
	\subsection{Verilog关键字}
	初学者常用的关键字有,module/endmodule、 input、 output、 wire、 reg、assign、always、 initial、 begin/end、 posedge、 negedge、 if、 else、case、 endcase。\par
	其中,本人较为不熟悉的是case/endcase语句,故总结如下:具有相同优先级的多路条件分支,两个关键字必须成对出现。一般出现在always的过程语句部分,而不能在模块内部单独出现。例如:\par
	\begin{lstlisting}[language=Verilog]
	module test(
		input wire a,b,clk,
		output reg o);
		
	always@( posedge clk )
	case({a,b})
		2’b00: o <= 1’b0;2’b01: o <= 1’b0;
		2’b10: o <= 1’b0;
		2’b11: o <= 1’b1;
	endcase
	endmodule
	\end{lstlisting}
	
	\subsection{Verilog基本结构}
	Verilog基本结构可以总结为如下:
	\begin{lstlisting}[language=Verilog]
	module 模块名(
		输入端口定义 //输入端口只能是 wire 类型
		输出端口定义 //输入信号可根据需要定义成 wire 或 reg 类型
		);
		内部线信号定义 //内部信号可根据需要定义成 wire 或 reg 类型
		模块实例化 //实例化的输出端只能接 wire 类型信号
		assign 连续赋值语句
		always 过程语句
	endmodule
	\end{lstlisting}
	
	\subsection{Verilog 数据及类型}
	Verilog有四种基本的值:0, 1, x, z。\par
	Verilog有三种常量:整数,实数,字符串。其中整数例如6'b101100。可选的有二进制(b/B)、八进制(o/O)、十进制(d/D)和十六进制(h/H)
	Verilog有两种常用数据类型:wire(线网类型)和reg(寄存器类型)关于两种数据类型的使用只需要遵循以下规则即可:凡是通过 assign 语句赋值的信号(一定是组合逻辑赋值信号),都应定义成 wire 类型,凡是在 always 语句中赋值的信号(可能是组合逻辑赋值信号、也可能是时序逻辑赋值信号),都应定义成 reg 类型。
	
	
	\subsection{Verilog 操作符}
	\begin{lstlisting}[language=Verilog]
	算数运算符: +、-、*、/、%
	关系运算符:>、<、>=、<=、==、!=
	逻辑操作符:&&、||、!、~、&、|、^ 、~^
	归约操作符:&、~&、|、~|、^、~^
	条件操作符:? : //三目运算符
	移位操作符:<<、>>
	拼接操作符:{}
	\end{lstlisting}
	
	
	\subsection{Verilog 表达式}
	表达式可以是以下类型:常数、参数、线网、寄存器、位选择、部分选择、存储器单元、函数调用。\par
	表达式都有一个值, 因此可以将其赋给 wire 或 reg 类型的信号,也可以用在逻辑判断语句(如 if、 case)中。 如 assign a = 表达式 1、always@(*) if(表达式)... else ...。
	
	
	
	\subsection{模块调用}
	模块实例语句的形式为:模块名 实例化名(端口关联)。\par
	端口信号可以通过位置或名称进行关联,但两种关联方式不能混用。 
	\begin{lstlisting}[language=Verilog]
	add add_inst1(a,b,s,carry1);//通过位置关联
	add add_inst2(.a(s),.b(cin),.sum(sum),.cout(cout));//通过名称关联
	\end{lstlisting}
	
	
	
	\subsection{代码实例}
	\begin{lstlisting}[language=Verilog]
	8bit 位宽 4 选 1 选择器,纯组合逻辑
	module mux_4to1( //8bit 位宽的 4 选 1 选择器
		input [7:0] a,b,c,d,
		input [1:0] sel,
		output reg [7:0] o); //always 语句内赋值的信号都应定义成 reg 类型
		
	always@(*) //always 语句内实现组合逻辑
	begin
		case(sel)
			2’b00: o = a; //组合逻辑使用“=”进行赋值
			2’b01: o = b;
			2’b10: o = c;
			2’b11: o = d;
			default: o = 8’h0;//用 case 语句实现组合逻辑时一定要有 default
		endcase
	end
	endmodule
	================================
	1~10 循环计数的计数器
	module cnt_1to10(
		input clk,rst_n,
		output reg [3:0] cnt);
		
	always@(posedge clk or negedge rst_n)
	//时序控制条件为时钟上升沿和复位的下降沿
	begin
		if(!rst_n) //复位信号优先级最高,应是第一个判断的条件
			cnt <= 4’h1;
		else if(cnt>=10)
			cnt <= 4’h1;
		else
			cnt <= cnt + 4’h1;
	end
	endmodule
	\end{lstlisting}
	
	
	\section{实验练习}
	\subsection{题目1.}
	修改结果如下:
	\begin{lstlisting}[language = Verilog]
	module test(
		input a,
		output reg b); //因为b需要在always内赋值,须为reg类型
		
	// if-else语句不能直接在模块内使用,
	// 应在always语句的过程语句部分内使用
	always @(*)
	begin
		if(a) b = 1’b0;
		else b = 1’b1;
	end
	endmodule
	\end{lstlisting}
	
	
	
	\subsection{题目2.}
	\begin{lstlisting}[language = Verilog]
	module test(
		input [4:0] a,
		output reg[4:0] b);
		
	always@(*)
		b = a;
	endmodule
	\end{lstlisting}
	
	
	
	\subsection{题目3.}
	列表如下:\par
	\begin{tabular}{|c|c|}
		\hline 
		c & 8'b0011 0000 \\ 
		\hline 
		d & 8'b1111 0011 \\ 
		\hline 
		e &  8'b1100 0011\\ 
		\hline 
		f &  8'b1100 1100\\ 
		\hline 
		g &  8'b0011 0000\\ 
		\hline 
		h &  8'b0000 0110\\ 
		\hline 
		i &  8'b0000 0000\\ 
		\hline 
		j &  8'b1111 0000\\ 
		\hline 
		k &  8'b0100 0011\\ 
		\hline 
	\end{tabular} 



	\subsection{题目4.}
	\begin{lstlisting}[language = Verilog]
	module sub_test(
		input a,b,output wire c); //将reg修改为wire,因为通过assign赋值
		
	assign c = (a<b)? a : b;
	endmodule
	
	module test(
		input a,b,c,
		output o);
		
	reg temp;
	sub_test(a, b, temp); //两种传参方法不能同时使用
	sub_test(temp, c, o); //两种传参方法不能同时使用
	endmodule
	\end{lstlisting}
	
	
	
	
	\subsection{题目5.}
	\begin{lstlisting}[language = Verilog]
	module sub_test(
		input a,b;
		output o); //output的声明应该也要在括号内
		
	assign o = a + b;
	endmodule
	
	
	module test(
		input a,b,
		output c); //在always内赋值,应是reg类型
	
	//always里不应该使用模块的例化
	sub_test sub_test(a, b, c);
	endmodule
	\end{lstlisting}
	
	
	\section{总结与思考}
	
	\subsection{本次实验的收获}
	在本次实验中,我系统了解了Verilog这个语言,也发现了一些以往的误区,收获颇丰。\par
	
	\subsection{评价本次实验的难易程度}
	本次实验内容难度适中。\par
	
	\subsection{评价本次实验的任务量}
	本次实验任务量合理。\par
	
	\subsection{为本次实验提供改进建议}
	暂无建议。
	
\end{document}
