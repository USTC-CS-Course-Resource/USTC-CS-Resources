\documentclass[UTF8]{article}
\usepackage{graphicx}
\usepackage{subfigure}
\usepackage{amsmath}
\usepackage{makecell}
\usepackage[utf8]{inputenc}
\usepackage[space]{ctex} %中文包
\usepackage{listings} %放代码
\usepackage{xcolor} %代码着色宏包
\usepackage{CJK} %显示中文宏包
\usepackage{float}
\usepackage{makecell}
\usepackage{diagbox}
\usepackage{bm}
\usepackage{ulem} 
\usepackage{amssymb}
\usepackage{soul}
\usepackage{color}
\usepackage{geometry}
\usepackage{fancybox} %花里胡哨的盒子
\usepackage{xhfill} %填充包, 可画分割线 https://www.latexstudio.net/archives/8245
\usepackage{multicol} %多栏包
\usepackage{enumerate} %可以方便地自定义枚举标题
\usepackage{multirow} %表格中多行单元格合并
\usepackage{wasysym} %可以使用wasysym里的一堆奇奇怪怪的符号
\definecolor{mygreen}{rgb}{0,0.6,0}
\definecolor{mygray}{rgb}{0.5,0.5,0.5}
\definecolor{mymauve}{rgb}{0.58,0,0.82}
\lstset{
	backgroundcolor=\color{white}, 
	basicstyle = \footnotesize,       
	breakatwhitespace = false,        
	breaklines = true,                 
	captionpos = b,                    
	commentstyle = \color{mygreen}\bfseries,
	extendedchars = false,             
	frame =shadowbox, 
	framerule=0.5pt,
	keepspaces=true,
	keywordstyle=\color{blue}\bfseries, % keyword style
	language = C++,                     % the language of code
	otherkeywords={string}, 
	numbers=left, 
	numbersep=5pt,
	numberstyle=\tiny\color{mygray},
	rulecolor=\color{black},         
	showspaces=false,  
	showstringspaces=false, 
	showtabs=false,    
	stepnumber=1,         
	stringstyle=\color{mymauve},        % string literal style
	tabsize=4,          
	title=\lstname                      
}


\title{计算方法编程练习:级数求和}
\author{王章瀚 PB18111697}


\begin{document}

\maketitle


\section{Introduction}
本次作业要求在给定的x下求以下级数的近似值:
$$\psi (x)=\sum_{k=1}^{+\infty}\frac{1}{k(k+x)}$$

其中分别取$x=0.0,\ 0.5,\ 1.0,\ \sqrt{2},\ 10.0,\ 100.0,\ 300.0$,计算$\psi(x)$的近似值,要求截断误差在$10^{-6}$内。

输出格式:$x\mbox{和}\psi (x)\mbox{的值}$

\qquad $x=0.0, y=1.644934066848226$

\qquad $x=0.5, y=1.227411277760219$

\section{Method}
\subsection{Method 1}
由于题目要求的是近似值,且要求限制是截断误差。故应先计算,在截断误差的要求范围内,k应算到哪一步。

对于这个函数,由于题目要求的$x$满足$x\ge 0$,故对较大的$k$有:
\begin{align*}
E(x)&=\sum_{k=MIN\_K+1}^{+\infty}\frac{1}{k(k+x)}\\
&\leq \sum_{k=MIN\_K+1}^{+\infty}\frac{1}{k^{2}}\\
&\leq \int_{MIN\_K}^{+\infty}\frac{1}{k^{2}}\,dk\\
&=\frac{1}{MIN\_K}\leq 10^{-6}
\end{align*}
因此有:$$MIN\_K \ge \frac{1}{10^{-6}} = 10^{6}$$

该方法的实现代码及效果放在了附录A.

\subsection{Method 2}
此外,根据课程主页上原题的提示,还有优化后的方法,即注意到$\psi (1)=1$的事实,可以考虑求$$\frac{\psi(x)-\psi(1)}{1-x}=\sum_{k=1}^{+\infty}\frac{1}{k(k+1)(k+x)}$$
易见,这个函数相较于原本的$\psi(x)$收敛速度更快。
考虑到$0\leq x\leq 300$类似地,可以算出原函数的截断误差$E(x)$满足,
\begin{align*}
E(x)&\leq 300\sum_{k=MIN\_K+1}^{+\infty}\frac{1}{k(k+1)(k+x)}\\
&\leq \sum_{k=MIN\_K+1}^{+\infty}\frac{300}{k^{3}}\\
&\leq \int_{MIN\_K}^{+\infty}\frac{300}{k^{3}}\,dk\\
&=\frac{300}{2MIN\_K^{2}}\leq 10^{-6}
\end{align*}
因此有:$$MIN\_K \ge \sqrt{\frac{150}{10^{-6}}}$$
这样的构造使得计算量大大减少。
该方法的实现代码及效果放在了附录B.

\section{Result}
按照Method1的算法,其输出结果如下:\par
\qquad x=0.000000,y=1.644933066848770e+00\par
\qquad x=0.500000,y=1.227410277760964e+00\par
\qquad x=1.000000,y=9.999990000010476e-01\par
\qquad x=1.414214,y=8.749819960221313e-01\par
\qquad x=10.000000,y=2.928958254023105e-01\par
\qquad x=100.000000,y=5.187277522689390e-02\par
\qquad x=300.000000,y=2.094121308480047e-02\par
\vspace{1cm}
按照Method2的算法,其输出结果如下:\par
\qquad x=0.000000,y=1.644934066826041e+00\par
\qquad x=0.500000,y=1.227411277749111e+00\par
\qquad x=1.000000,y=1.000000000000000e+00\par
\qquad x=1.414214,y=8.749829960301527e-01\par
\qquad x=10.000000,y=2.928968255968175e-01\par
\qquad x=100.000000,y=5.187377737541299e-02\par
\qquad x=300.000000,y=2.094221956986198e-02\par

从上述答案中可见,尽管使用Method1的算法在截断误差要求为$10^{-6}$的情况下,也能满足截断误差在$10^{-6}$范围内的要求,但如果提高截断误差精确度,则计算过于缓慢。相比之下Method2的算法,能够更快地得到计算结果。
\section{Discussion}
本题中,由于要求的截断误差不那么严格,因此暴力计算的方法并不会对效率造成太大影响。但如果要求精度更高,则需要考虑对要求的函数稍作修改,从而提高收敛速度等。

\section*{A Computer Code of Method 1}
\begin{lstlisting}[language=C++]
#include <cstdio>
#include <cmath>

// 设置允许截断误差值
const double TRUNCATION_ERROR = 1e-6;
// 原方法下,计算在允许截断误差下的k应计算到多少位
const double MIN_K_0 = 1/TRUNCATION_ERROR;

// x们及x的个数
const double array_x[] = { 0.0, 0.5, 1.0, pow(2, 0.5), 10.0, 100.0, 300.0 };
const int X_NUM = 7;

// 原方法
double psi_0(double x) {
	double answer = 0;
	double k = 1;
	// 直接对1/((x+k)x)求和
	while (k <= MIN_K_0) {
		answer += 1/(k*(k+x));
		k++;
	}
	return answer;
}

int main()
{
	for (double i : array_x) {
		printf("x=%lf,y=%.15e\n", i, psi_0(i));
	}
}
\end{lstlisting}

\section*{B Computer Code of Method 2}
\begin{lstlisting}[language=C++]
#include <cstdio>
#include <cmath>
using namespace std;

// 设置允许截断误差值
const double TRUNCATION_ERROR = 1e-6;
// 新方法下,计算在允许截断误差下的k应计算到多少位
const double MIN_K = 150/pow(TRUNCATION_ERROR, 0.5);
// 原方法下,计算在允许截断误差下的k应计算到多少位
const double MIN_K_0 = 1/TRUNCATION_ERROR;

// x们及x的个数
const double array_x[] = { 0.0, 0.5, 1.0, pow(2, 0.5), 10.0, 100.0, 300.0 };
const int X_NUM = 7;

// 经提示改进的方法
double psi(double x) {
	double answer = 0;
	double k = 1;
	// 先计算(psi(x)-psi(1))/(1-x),其收敛速度更快
	while (k <= MIN_K) {
		answer += 1/(k*(k+x)*(k+1));
		k++;
	}
	return 1 + (answer)*(1 - x);
}

int main()
{
	for (double i : array_x) {
		printf("x=%lf,y=%.15e\n", i, psi_0(i));
	}
	printf("\n");
	
	for (double i : array_x) {
		printf("x=%lf,y=%.15e\n", i, psi(i));
	}
}

\end{lstlisting}

\end{document}
